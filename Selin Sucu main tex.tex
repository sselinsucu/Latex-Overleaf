\documentclass[10pt]{article}
\usepackage{graphicx} % Required for inserting images

\usepackage{titlesec}
\usepackage{xcolor}
\usepackage{amsthm}
\usepackage{amsmath}






\begin{document}

\chapter\Large\textbf{Chapter 1} \\

\begin{huge}
    \textbf{Power Series}
\end{huge} \\

\begin{normalsize}
\textbf{Definition 1.} {A} \textbf{power series}
{is a series of the form} 
\begin{equation}
\sum_{n=0}^{\infty}= {c_nx^n}={c_0+ c_1x+ c_2x+...}
\end{equation}
{where $x$ is a variable and the $c_n$'s are constants called the} \textbf{coefficients} {of the
series.} \end{normalsize} 
\begin{par}
    \begin{normalsize}
   {For each fixed $x$, the series (1) is a series of constants that we can test for
convergence or divergence. A power series may converge for some values of $x$
and diverge for other values of $x$.}
\end{normalsize}
\end{par} \\

\begin{par}
\begin{normalsize}
 \noindent \textcolor{red}{\textit{Example 1.}}
 {For what values of $x$ is the series} ${\sum_{n=0}^{\infty}{n!x^n}}$ {convergent?}\\
 
 \end{normalsize}
\end{par}
    \begin{normalsize}
    \noindent {\textit{Solution.}}
    {We use the Ratio Test. If we let $a_n$, as usual, denote the  \(n\)th term of the series, then $a_n=n!x^n$. If {$x\neq 0$}, we have}
        \end{normalsize}
        \begin{normalsize}
        
\begin{center} $ \lim_{n \to \infty} \left|\frac{a_{n+1}}{a_n}\right|=   \lim_{n \to \infty}\left|\frac{(n+1)!x^{n+1}}{n!x^n}\right|= \lim_{n \to \infty}(n + 1)|x|= \infty.$ \end{center}

\end{normalsize}
\begin{normalsize}
\noindent
{By the Ratio Test, the series diverges when {$x\neq 0$}. Thus, the given series
converges only when x = 0.}\\
\qed

 \noindent \textcolor{red}{\textit{Example 2.}}
{For what values of $x$ does the series $\sum_{n=1}^{\infty}\frac{(x-3)^n}{n}$ converge?}\end{normalsize} \\

\begin{normalsize}
     \noindent \textit{Solution.}
     {Let} $a_n= \frac{(x-3)^n}{n}$. {Then}
     
     \begin{center} $\left|\frac{a_{n+1}}{a_n}\right|=\left|\frac{(x-3)^{n+1}}{n+1} \frac{n}{{(x-3)}^n}\right|$\end{center}
            
   \begin{center}
       
   $=\frac{1}{1+\frac{1}{n}}\left|{x-3}\right| \to\left|{x-3}\right| $ {as}
    $n\to \infty $ 
    \end{center}
\end{normalsize}

    \newpage
    \begin{normalsize}
        {By the Ratio Test, the given series is absolutely convergent, and therefore convergent when $\left|{x-3}\right| < 1$, and divergent when $\left|{x-3}\right| > 1$. Now}
        \begin{center}
            $\left|{x-3}\right| < 1 \iff -1<x-3<1 \iff 2<x<4$
        \end{center}
       {so the series converges when $2 < x < 4$, and diverges if $x < 2$ or $x > 4$. The Ratio Test gives no information when  $\left|{x-3}\right|=1$ so we must consider $x = 2$ and $x = 4$ separately. If we put $x = 4$ in the series, it becomes $\sum\frac{1}{n}$, the harmonic series, which is divergent. If $x = 2$, the series is $\sum\frac{{-1}^n}{n}$ , which converges by the Alternating Series Test. Thus, the given power series converges for $2\leq x < 4$.}
    \end{normalsize}
    \qed\\
    
    \begin{normalsize}
    \noindent
\textbf{Theorem 1.}  
\textit{For a given power series there are only three possibilities:}\\

    \textit{1. The series converges only when $x = a$} \\
    
    \textit{2. The series converges for all $x$.}\\
    
    \textit{3. There is a positive number such that the series converges if $\left|{x-a}\right| < R$ and diverges if $\left|{x-a}\right| > R$.}\\
     \end{normalsize}
     \begin{normalsize}
     \par
     {The number $R$ in case (3) is called the \textbf{ radius of convergence} of the power series. By convention, the radius of convergence is $R = 0$ in case (1) and $R = \infty$ in case (2). The \textbf{interval of convergence} of a power series is the interval that consists of all values of $x$ for which the series converges. In case (1) the interval consists of just a single point $a$. In case (2) the interval is  $(-\infty,\infty)$. In case (3) note that the inequality $\left|{x-a}\right|< R$ can be rewritten as $a-R < x < a + R$. When $x$ is an \textbf{endpoint} of the interval, that is, $x =a \pm R$, anything can happen; the series might converge at one or both endpoints or it might diverge at both endpoints. Thus, in case (3) there are four possibilities for the interval of convergence: }
     \begin{center}
         $(a-R,a+R)$, $(a-R,a+R]$, $[a-R,a+R)$ {or} $[a-R,a+R]$
         
     \end{center}
     \begin{center}
         \begin{tabular}{|c|c|c|c|}
\hline
Series & Radius Of Convergence & Interval of Convergence \\
\hline
$\sum_{n=0}^{\infty}x^n$ & $R=1$ & $(-1,1)$ \\
\hline
$\sum_{n=0}^{\infty}n!x^n$ & $R=0$ & $\{0\}$ \\

\hline
$\sum_{n=0}^{\infty}\frac{(x-3)^n}{n}$ & $R=1$ & $[2,4)$ \\
\hline
\end{tabular}
     \end{center}
     \end{normalsize}
     
        
   
   
    
\end{document}


